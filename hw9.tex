\documentclass[fleqn]{article}
\oddsidemargin 0.0in
\textwidth 6.0in
\thispagestyle{empty}
\usepackage{import}
\usepackage{amsmath}
\usepackage{graphicx}
\usepackage[english]{babel}
\usepackage[utf8x]{inputenc}
\usepackage{float}
\usepackage[colorinlistoftodos]{todonotes}

\definecolor{hwColor}{HTML}{AD53BA}

\begin{document}

  \begin{titlepage}

    \newcommand{\HRule}{\rule{\linewidth}{0.5mm}} % Defines a new command for the horizontal lines, change thickness here

    \center % Center everything on the page
    

    \textsc{\LARGE Arizona State University}\\[1.5cm] % Name of your university/college

    \textsc{\LARGE Physics III }\\[1.5cm] % Major heading such as course name


    \begin{figure}
      \includegraphics[width=\linewidth]{asu.png}
    \end{figure}


    \HRule \\[0.4cm]
    { \huge \bfseries Homework 9}\\[0.4cm] 
    \HRule \\[1.5cm]
    
    \textbf{Behnam Amiri}

    \bigbreak

    \textbf{Prof: David Smith}

    \bigbreak


    \textbf{{\large \today}\\[2cm]}

    \vfill % Fill the rest of the page with whitespace

  \end{titlepage}

  \begin{enumerate}
    \item (64) What is the smallest Bragg angle for x rays of wavelength 30pm to reflect from reflecting planes spaced 0.30 nm apart in a calcite crystal?

      \textcolor{hwColor}{ 
        $ 
          2dsin(\theta)=m \lambda \rightarrow \theta=arcsin(\dfrac{m \lambda}{2d})=arcsin\left(\dfrac{30 \times 10^{-12}}{2(0.30 \times 10^{-9})}\right) \\
          \Longrightarrow \theta_{min}=2.86^\circ
        $ 
      }

    \item (66) An x-ray beam of a certain wavelength is incident on a NaCl crystal, at 30.0° to a certain family of reflecting planes of spacing 39.8 pm. If the reflection from those planes is of the first order, what is the wavelength of the x rays?

      \textcolor{hwColor}{ 
        $ 
          \lambda=\dfrac{2dsin(\theta)}{n}=\dfrac{2(39.8 \times 10^{-12} sin(30))}{1} \\
          \Longrightarrow \lambda=3.98 \times 10^{-11} ~~~ m
        $ 
      }

    \item (67) Figure 36-46 is a graph of intensity versus angular position $\theta$ for the diffraction of an x-ray beam by a crystal. The horizontal scale is set by $\theta=2.00°$. The beam consists of two wavelengths, and the spacing between the reflecting planes is 0.94 nm. What are the (a) shorter and (b) longer wavelengths in the beam?

    \textcolor{hwColor}{ 
      $ 
        a. \\ 
        2d~sin(\theta)=m \lambda \rightarrow \lambda_1=\dfrac{2dsin(\theta)}{m}=\dfrac{2(0.94 \times 10^{-9}sin(0.75))}{1}=2.46 \times 10^{-11} ~~ m
      $ 
    }

    \textcolor{hwColor}{   
      \rule{15cm}{0.4pt}   
    } 

    \textcolor{hwColor}{ 
      $ 
        b. \\ 
        2d~sin(\theta)=m \lambda \rightarrow \lambda_2=\dfrac{2dsin(\theta)}{m}=\dfrac{2(0.94 \times 10^{-9}sin(1.15))}{1}=3.77 \times 10^{-11} ~~ m
      $ 
    }

    \item (70) In Fig. 36-47, first-order reflection from the reflection planes shown occurs when an x-ray beam of wavelength 0.260 nm makes an angle $\theta=63.8°$ with the top face of the crystal. What is the unit cell size $a_0$?

      \textcolor{hwColor}{ 
        $
          \theta=63.8-45=18.8^\circ \\
          sin(45)=\dfrac{d}{a_0} \rightarrow d=\dfrac{a_0}{\sqrt{2}} \\
        $ \\
        Bragg's equation: \\
        $
          2d~sin(\theta)=m \lambda \rightarrow d=\dfrac{m \lambda}{2~sin(\theta)} \\
          \Longrightarrow a_0=\dfrac{m \sqrt{2} \lambda}{2~sin(\theta)}=\dfrac{1 \times \sqrt{2} \times 0.260 \times 10^{-9}}{2 sin(18.8)}=5.704 \times 10^{-10} ~~ m
        $
      }

    \item (77) In a single-slit diffraction experiment, there is a minimum of intensity for orange light ($\lambda=600 nm$) and a minimum of intensity for blue-green light ($\lambda=500 nm$) at the same angle of 1.00 mrad. For what minimum slit width is this possible?

      \textcolor{hwColor}{ 
        $
          d~sin(\theta)=m_1 \lambda_1 \\
          d~sin(\theta)=m_2 \lambda_2 \\
          m_1 \lambda_1=m_2 \lambda_2 \Rightarrow m_1 \times 600~nm=m_2 \times 500~nm \\
          \begin{cases}
            m_1=5 ~~~~ min \\
            m_2=6 ~~~~ min \\
          \end{cases} \\
          d~sin(\theta)=m_1 \lambda_1 \Rightarrow d=\dfrac{m_1 \lambda_1}{sin(\theta)}=\dfrac{5 \times 600 \times 10^{-9} ~~ m}{10^-3 ~~ rad} \\
          d=3 ~~ mm
        $
      }


    \item (84) In a two-slit interference pattern, what is the ratio of slit separation to slit width if there are 17 bright fringes within the central diffraction envelope and the diffraction minima coincide with two slit interference maxima?

      \textcolor{hwColor}{ 
        $
          a~sin(\theta)=1 \times \lambda \\
          d~sin(\theta)=m_2 \lambda \\
          \Longrightarrow m_2=\dfrac{d}{a} \\
          \\
          m_2=\dfrac{19-1}{2}=9
        $
      }

    \item (96) When monochromatic light is incident on a slit $22.0 \mu m$ wide, the first diffraction minimum lies at 1.80° from the direction of the incident light.What is the wavelength?

      \textcolor{hwColor}{ 
        $
          d~sin(\theta)=m \lambda \rightarrow \lambda=\dfrac{d sin(\theta)}{m} \\
          \\
          \lambda=\dfrac{22 \times 10^{-6} sin(\dfrac{1.80 \times 2\pi}{360})}{1}=1.206 \times 10^{-8} m
        $
      }

  \end{enumerate}

\end{document}
