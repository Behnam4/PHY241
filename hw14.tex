\documentclass[fleqn]{article}
\oddsidemargin 0.0in
\textwidth 6.0in
\thispagestyle{empty}
\usepackage{import}
\usepackage{amsmath}
\usepackage{graphicx}
\usepackage[english]{babel}
\usepackage[utf8x]{inputenc}
\usepackage{float}
\usepackage[colorinlistoftodos]{todonotes}

\definecolor{hwColor}{HTML}{AD53BA}

\begin{document}

  \begin{titlepage}

    \newcommand{\HRule}{\rule{\linewidth}{0.5mm}} % Defines a new command for the horizontal lines, change thickness here

    \center % Center everything on the page


    \textsc{\LARGE Arizona State University}\\[1.5cm] % Name of your university/college

    \textsc{\LARGE Mathematical Methods For Physics I }\\[1.5cm] % Major heading such as course name


    \begin{figure}
      \includegraphics[width=\linewidth]{asu.png}
    \end{figure}


    \HRule \\[0.4cm]
    { \huge \bfseries Homework 14}\\[0.4cm] 
    \HRule \\[1.5cm]

    \textbf{Behnam Amiri}

    \bigbreak

    \textbf{Prof: David Smith}

    \bigbreak


    \textbf{{\large \today}\\[2cm]}

    \vfill % Fill the rest of the page with whitespace

  \end{titlepage}

  \textbf{Chapter 39}
  \begin{enumerate}
    \item (50) Light of wavelength 102.6 nm is emitted by a hydrogen
    atom. What are the (a) higher quantum number and (b) lower
    quantum number of the transition producing this emission? (c)
    What is the name of the series that includes the transition?

      \textcolor{hwColor}{
        A: \\
        $
          E=A(\dfrac{1}{n^2_2}-\dfrac{1}{n^2_1}) \rightarrow \dfrac{E}{hc}=\dfrac{A}{hc}(\dfrac{1}{n^2_2}-\dfrac{1}{n^2_1}) \\
          \\
          \dfrac{hc}{\lambda A}=\dfrac{1}{n^2_2}-\dfrac{1}{n^2_1} \\
          \\
          \dfrac{(6.626 \times 10^{-34})(2.998 \times 10^8)}{(102.6 \times 10^{-9})(13.6 \times 1.602 \times 10^{-19})}=\dfrac{1}{n^2_2}-\dfrac{1}{n^2_1} \\
          \\
        $
        The higher quantum number is: $n_1=3$ \\
        \\
      }

      \textcolor{hwColor}{ 
        \rule{16cm}{0.6pt} 
      } 

      \textcolor{hwColor}{
        B: \\
        $
          n_2=1 \\
          \\
        $
      }

      \textcolor{hwColor}{ 
        \rule{16cm}{0.6pt} 
      } 

      \textcolor{hwColor}{
        C: \\
        The transition happens when the electron goes from $n_1$ to $n_2$.
      }


    \item (52) A hydrogen atom is excited from its ground state to the state
    with n=4. (a) How much energy must be absorbed by the atom?
    Consider the photon energies that can be emitted by the atom as it
    de-excites to the ground state in the several possible ways. (b) How
    many different energies are possible; what are the (c) highest, (d)
    second highest, (e) third highest, (f) lowest, (g) second lowest, and
    (h) third lowest energies?

    \textcolor{hwColor}{
      A: \\
      $
        E=A(\dfrac{1}{n^2_2}-\dfrac{1}{n^2_1})=(13.6)(\dfrac{1}{1}-\dfrac{1}{4^2}) \\
        \\
        E=12.75 ~~ eV
      $
    }

    \textcolor{hwColor}{ 
      \rule{16cm}{0.6pt} 
    } 

    \textcolor{hwColor}{
      B: \\
      $
        4 \rightarrow 3 ~~~~ 4 \rightarrow 2 ~~~~ 4 \rightarrow 1 ~~~~ 3 \rightarrow 2 ~~~~ 3 \rightarrow 1 ~~~~ 2 \rightarrow 1 \\
      $ \\
      There are six different ways.
    }

    \textcolor{hwColor}{ 
      \rule{16cm}{0.6pt} 
    } 

    \textcolor{hwColor}{
      C: \\
      $
        E_{4 \rightarrow 1}=(13.6)(\dfrac{1}{1^2}-\dfrac{1}{}) \\
        \\
        E_{4 \rightarrow 1}=12.75 ~~ eV
      $
    }

    \textcolor{hwColor}{ 
      \rule{16cm}{0.6pt} 
    } 

    \textcolor{hwColor}{
      D: \\
      $
        E_{3 \rightarrow 1}=(13.6)(\dfrac{1}{1^2}-\dfrac{1}{3^2}) \\
        \\
        E_{3 \rightarrow 1}=12.1 ~~ eV
      $
    }

    \textcolor{hwColor}{ 
      \rule{16cm}{0.6pt} 
    } 

    \textcolor{hwColor}{
      E: \\
      $
        E_{2 \rightarrow 1}=(13.6)(\dfrac{1}{1^2}-\dfrac{1}{2^2}) \\
        \\
        E_{2 \rightarrow 1}=10.2 ~~ eV
      $
    }

    \textcolor{hwColor}{ 
      \rule{16cm}{0.6pt} 
    } 

    \textcolor{hwColor}{
      F: \\
      $
        E_{4 \rightarrow 3}=(13.6)(\dfrac{1}{3^2}-\dfrac{1}{4^2}) \\
        \\
        E_{4 \rightarrow 3}=0.661 ~~ eV
      $
    }

    \textcolor{hwColor}{ 
      \rule{16cm}{0.6pt} 
    } 

    \textcolor{hwColor}{
      G: \\
      $
        E_{3 \rightarrow 2}=(13.6)(\dfrac{1}{2^2}-\dfrac{1}{3^2}) \\
        \\
        E_{3 \rightarrow 2}=1.89 ~~ eV
      $
    }

    \textcolor{hwColor}{ 
      \rule{16cm}{0.6pt} 
    } 

    \textcolor{hwColor}{
      H: \\
      $
        E_{4 \rightarrow 2}=(13.6)(\dfrac{1}{2^2}-\dfrac{1}{4^2}) \\
        \\
        E_{4 \rightarrow 2}=2.55 ~~ eV
      $
    }
    
    
    \item (57) An electron is trapped in a one-dimensional infinite potential
    well. Show that the energy difference $\Delta E$ between its quantum levels n and $n+2$ is $\dfrac{h^2}{2mL^2}(n+1)$.

    \textcolor{hwColor}{
      A: \\
      $
        E_n=(\dfrac{h^2}{8mL^2})n^2 \\
        \\
        E_1=(\dfrac{h^2}{8mL^2}) \\
        \\
        E_n=E_1n^2 ~~~~~ E_{n+1}=E_1(n+1)^2 \\
        \\
        \\
        E_{n+1}-E_n=E_1(n+1)^2-E_1n^2=(2n+1)E_1 \\
        \\
        \Delta E_{34}=E_4-E_3=(6+1)E_1=7E_1 \\
        \\
        21E_1=(2n+1)E_1 \rightarrow n=10
      $ So the higher quantum number is 11.
    }

    \textcolor{hwColor}{ 
      \rule{16cm}{0.6pt} 
    } 

    \textcolor{hwColor}{
      B: \\
      $
        n=10
      $ 
    }

    \textcolor{hwColor}{ 
      \rule{16cm}{0.6pt} 
    } 

    \textcolor{hwColor}{
      C: \\
      $
        14E_1=(2n+1)E_1 \\
        \\
        14=2n+1
      $ 
    }


  \end{enumerate}

  \textbf{Chapter 40}
  \begin{enumerate}
    \item (30)


    \item (53)
    
    
    \item (55)
  \end{enumerate}

  \textbf{Chapter 41}
  \begin{enumerate}
    \item (3)


    \item (6)
    
    
  \end{enumerate}

\end{document}
