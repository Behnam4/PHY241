\documentclass[fleqn]{article}
\oddsidemargin 0.0in
\textwidth 6.0in
\thispagestyle{empty}
\usepackage{import}
\usepackage{amsmath}
\usepackage{graphicx}
\usepackage[english]{babel}
\usepackage[utf8x]{inputenc}
\usepackage{float}
\usepackage[colorinlistoftodos]{todonotes}

\definecolor{hwColor}{HTML}{AD53BA}

\begin{document}

  \begin{titlepage}

    \newcommand{\HRule}{\rule{\linewidth}{0.5mm}} % Defines a new command for the horizontal lines, change thickness here

    \center % Center everything on the page



    \textsc{\LARGE Arizona State University}\\[1.5cm] % Name of your university/college

    \textsc{\LARGE Physics III }\\[1.5cm] % Major heading such as course name


    \begin{figure}
      \includegraphics[width=\linewidth]{asu.png}
    \end{figure}


    \HRule \\[0.4cm]
    { \huge \bfseries Homework 12}\\[0.4cm] 
    \HRule \\[1.5cm]

    \textbf{Behnam Amiri}

    \bigbreak

    \textbf{Prof: David Smith}

    \bigbreak


    \textbf{{\large \today}\\[2cm]}

    \vfill % Fill the rest of the page with whitespace

  \end{titlepage}

  \begin{enumerate}
    \item (42) Calculate the de Broglie wavelength of (a) a 1.00 keV electron, (b) a 1.00 keV photon, and (c) a 1.00 keV neutron.

      \textcolor{hwColor}{ 
        (a): \\ 
        $ 
          \lambda = \dfrac{h}{p} ~~~~~ p=\sqrt{2mk} \\
          \\
          \longrightarrow \lambda=\dfrac{h}{\sqrt{2mk}}=\dfrac{6.626 \times 10^{-34}}{\sqrt{2(9.11 \times 10^{-31})(10^3 \times 1.602 \times 10^{-19})}} \\
          \\
          \lambda = 3.88 \times 10^{-11} ~~~ m
        $ 
      }

      \textcolor{hwColor}{ 
        (b): \\ 
        $ 
          \lambda = \dfrac{hc}{E}=\dfrac{(6.626 \times 10^{-34})(2.998 \times 10^8)(2.998 \times 10^8)}{1 \times 10^3 \times 1.602 \times 10^{-19}} \\
          \\
          \lambda = 1.24 \times 10^{-9} ~~~ m
        $ 
      }

      \textcolor{hwColor}{ 
        (c): \\ 
        $ 
          \lambda = \dfrac{6.626 \times 10^{-34}}{\sqrt{2(1.67 \times 10^{-34})(10^3 \times 1.602 \times 10^{-19})}} \\
          \\
          \lambda = 9.06 \times 10^{-13} ~~~ m
        $ 
      }


    \item (50) An electron and a photon each have a wavelength of 0.20 nm. What is the momentum (in kg.m/s) of the (a) electron and (b) photon? What is the energy (in eV) of the (c) electron and (d) photon?

      \textcolor{hwColor}{
        (a): \\
        $
          p=\dfrac{h}{\lambda}=\dfrac{6.63 \times 10^{-34}}{0.20 \times 10^{-9}} \Longrightarrow \lambda = 3.3 \times 10^{-24} ~~~ kg.m/s \\
        $ 
      }

      \textcolor{hwColor}{
        (b): \\
        $
          p=3.3 \times 10^{-24} ~~~ kg.m/s \\
        $
      }

      \textcolor{hwColor}{
        (c): \\
        $
          K_e=\dfrac{p^2}{1~m_e}=\dfrac{(3.3 \times 10^{-24})^2}{2(9.11 \times 10^{-31})}=38 ~~ eV \\
        $
      }

      \textcolor{hwColor}{
        (d): \\
        $
          K_{ph}=pc=(3.3 \times 10^{-24})(2.998 \times 10^8) \\
          \\
          K_{ph}=6.2 ~~ KeV
        $
      }

    \item (54) What are (a) the energy of a photon corresponding to wavelength 1.00 nm, (b) the kinetic energy of an electron with de
    Broglie wavelength 1.00 nm, (c) the energy of a photon corresponding to wavelength 1.00 fm, and (d) the kinetic energy of an
    electron with de Broglie wavelength 1.00 fm?

      \textcolor{hwColor}{
        (a): \\
        $
          E=\dfrac{hc}{\lambda}=\dfrac{(6.626 \times 10^{-34})(2.998 \times 10^8)}{10^{-9}} \\
          \\
          E=1240 ~~ eV \\
        $
      }

      \textcolor{hwColor}{
        (b): \\
        $
          K=\dfrac{p^2}{2m} ~~~~~~ p=\dfrac{h}{\lambda} \\
          \\
          K=\dfrac{h^2}{2m \lambda^2}=\dfrac{(6.626 \times 10^{-34})^2}{2(9.11 \times 10^{-31})(10^{-9})^2} \\
          \\
          K=1.50 ~~ eV \\
        $
      }

      \textcolor{hwColor}{
        (c): \\
        $
          E=\dfrac{hc}{\lambda}=\dfrac{(6.626 \times 10^{-34})(2.99 \times 10^8)}{10^{-15}} \\
          \\
          E=1.24 \times 10^9 ~~ eV \\
        $
      }

      \textcolor{hwColor}{
        (d): \\
        $
          K=\sqrt{(pc)^2+(mc^2)^2}-mc^2=\sqrt{(\dfrac{hc}{\lambda})^2+(mc^2)^2}-mc^2 \\
          \\
          K=\sqrt{(\dfrac{(6.626 \times 10^{-34})(2.99 \times 10^8)}{10^{-9}})^2+(6.704 \times 10^{-27})}-(8.188 \times 10^{-14}) \\
          \\
          K=1.25 \times 10^9 ~~ eV
        $
      }



    \item (59) Show that Eq. 38-17 is indeed a solution of Eq. 38-16 by substituting $\Psi(x)$ and its second derivative into Eq. 38-16 and noting
    that an identity results.


    \item (63) The uncertainty in the position of an electron along an x axis is given as 50 pm, which is about equal to the radius of a hydrogen atom.
    What is the least uncertainty in any simultaneous measurement of the
    momentum component px of this electron?


    \item (66) 


    \item (68) 


    \item (81) 


  \end{enumerate}

\end{document}
