\documentclass[fleqn]{article}
\oddsidemargin 0.0in
\textwidth 6.0in
\thispagestyle{empty}
\usepackage{import}
\usepackage{amsmath}
\usepackage{graphicx}
\usepackage[english]{babel}
\usepackage[utf8x]{inputenc}
\usepackage{float}
\usepackage[colorinlistoftodos]{todonotes}

\definecolor{hwColor}{HTML}{AD53BA}

\begin{document}

  \begin{titlepage}

    \newcommand{\HRule}{\rule{\linewidth}{0.5mm}} % Defines a new command for the horizontal lines, change thickness here

    \center % Center everything on the page



    \textsc{\LARGE Arizona State University}\\[1.5cm] % Name of your university/college

    \textsc{\LARGE Physics III }\\[1.5cm] % Major heading such as course name


    \begin{figure}
      \includegraphics[width=\linewidth]{asu.png}
    \end{figure}


    \HRule \\[0.4cm]
    { \huge \bfseries Homework 12}\\[0.4cm] 
    \HRule \\[1.5cm]

    \textbf{Behnam Amiri}

    \bigbreak

    \textbf{Prof: David Smith}

    \bigbreak


    \textbf{{\large \today}\\[2cm]}

    \vfill % Fill the rest of the page with whitespace

  \end{titlepage}

  \begin{enumerate}
    \item (42) Calculate the de Broglie wavelength of (a) a 1.00 keV electron, (b) a 1.00 keV photon, and (c) a 1.00 keV neutron.

      \textcolor{hwColor}{ 
        (a): \\ 
        $ 
          \lambda = \dfrac{h}{p} ~~~~~ p=\sqrt{2mk} \\
          \\
          \longrightarrow \lambda=\dfrac{h}{\sqrt{2mk}}=\dfrac{6.626 \times 10^{-34}}{\sqrt{2(9.11 \times 10^{-31})(10^3 \times 1.602 \times 10^{-19})}} \\
          \\
          \lambda = 3.88 \times 10^{-11} ~~~ m
        $ 
      }

      \textcolor{hwColor}{ 
        (b): \\ 
        $ 
          \lambda = \dfrac{hc}{E}=\dfrac{(6.626 \times 10^{-34})(2.998 \times 10^8)(2.998 \times 10^8)}{1 \times 10^3 \times 1.602 \times 10^{-19}} \\
          \\
          \lambda = 1.24 \times 10^{-9} ~~~ m
        $ 
      }

      \textcolor{hwColor}{ 
        (c): \\ 
        $ 
          \lambda = \dfrac{6.626 \times 10^{-34}}{\sqrt{2(1.67 \times 10^{-34})(10^3 \times 1.602 \times 10^{-19})}} \\
          \\
          \lambda = 9.06 \times 10^{-13} ~~~ m
        $ 
      }


    \item (50) An electron and a photon each have a wavelength of 0.20 nm. What is the momentum (in kg.m/s) of the (a) electron and (b) photon? What is the energy (in eV) of the (c) electron and (d) photon?

      \textcolor{hwColor}{
        (a): \\
        $
          p=\dfrac{h}{\lambda}=\dfrac{6.63 \times 10^{-34}}{0.20 \times 10^{-9}} \Longrightarrow \lambda = 3.3 \times 10^{-24} ~~~ kg.m/s \\
        $ 
      }

      \textcolor{hwColor}{
        (b): \\
        $
          p=3.3 \times 10^{-24} ~~~ kg.m/s \\
        $
      }

      \textcolor{hwColor}{
        (c): \\
        $
          K_e=\dfrac{p^2}{1~m_e}=\dfrac{(3.3 \times 10^{-24})^2}{2(9.11 \times 10^{-31})}=38 ~~ eV \\
        $
      }

      \textcolor{hwColor}{
        (d): \\
        $
          K_{ph}=pc=(3.3 \times 10^{-24})(2.998 \times 10^8) \\
          \\
          K_{ph}=6.2 ~~ KeV
        $
      }

    \item (54) What are (a) the energy of a photon corresponding to wavelength 1.00 nm, (b) the kinetic energy of an electron with de
    Broglie wavelength 1.00 nm, (c) the energy of a photon corresponding to wavelength 1.00 fm, and (d) the kinetic energy of an
    electron with de Broglie wavelength 1.00 fm?

      \textcolor{hwColor}{
        (a): \\
        $
          E=\dfrac{hc}{\lambda}=\dfrac{(6.626 \times 10^{-34})(2.998 \times 10^8)}{10^{-9}} \\
          \\
          E=1240 ~~ eV \\
        $
      }

      \textcolor{hwColor}{
        (b): \\
        $
          K=\dfrac{p^2}{2m} ~~~~~~ p=\dfrac{h}{\lambda} \\
          \\
          K=\dfrac{h^2}{2m \lambda^2}=\dfrac{(6.626 \times 10^{-34})^2}{2(9.11 \times 10^{-31})(10^{-9})^2} \\
          \\
          K=1.50 ~~ eV \\
        $
      }

      \textcolor{hwColor}{
        (c): \\
        $
          E=\dfrac{hc}{\lambda}=\dfrac{(6.626 \times 10^{-34})(2.99 \times 10^8)}{10^{-15}} \\
          \\
          E=1.24 \times 10^9 ~~ eV \\
        $
      }

      \textcolor{hwColor}{
        (d): \\
        $
          K=\sqrt{(pc)^2+(mc^2)^2}-mc^2=\sqrt{(\dfrac{hc}{\lambda})^2+(mc^2)^2}-mc^2 \\
          \\
          K=\sqrt{(\dfrac{(6.626 \times 10^{-34})(2.99 \times 10^8)}{10^{-9}})^2+(6.704 \times 10^{-27})}-(8.188 \times 10^{-14}) \\
          \\
          K=1.25 \times 10^9 ~~ eV
        $
      }

    \item (59) Show that Eq. 38-17 is indeed a solution of Eq. 38-16 by substituting $\Psi(x)$ and its second derivative into Eq. 38-16 and noting
    that an identity results.

      \textcolor{hwColor}{
        $
          \Psi(x)=Ae^{ikx}+Be^{-ikx} \longrightarrow \dfrac{d\Psi(x)}{dx}=A(ik)e^{ikx}-B(ik)e^{-ikx} \\
          \\
          \dfrac{d^2\Psi(x)}{dx^2}=A(ik)^2e^{ikx}+B(ik)^2e^{-ikx}=(ik)^2\left(Ae^{ikx}+Be^{-ikx}\right)=-k^2 \Psi(x) \\
          \\
        $
        The schrodinger wave equation: \\
        \\
        $
          \dfrac{d^2 \Psi(x)}{dx^2}+k^2 \Psi(x)=0 \\
          \\
          -k^2 \Psi(x)+k^2 \Psi(x)=0 \\
          \\
          \Longrightarrow \Psi(x)=Ae^{ikx}+Be^{-ikx}
        $
        is the general solution of $\dfrac{d^2 \Psi(x)}{dx^2}+k^2 \Psi(x)=0$
      }


    \item (63) The uncertainty in the position of an electron along an x axis is given as 50 pm, which is about equal to the radius of a hydrogen atom.
    What is the least uncertainty in any simultaneous measurement of the
    momentum component px of this electron?

      \textcolor{hwColor}{
        $
          \Delta p \approx \dfrac{\hbar}{\Delta x} \approx \dfrac{6.63 \times 10^{-34}~J.s}{2 \pi (50 ~ pm)} \\
          \\
          \Delta p \approx 2.1 \times 10^{-24} ~~ kg.m/s
        $
      }


    \item (66) Consider a potential energy barrier like that of Fig. 38-15 but whose height Ub is 6.0 eV and whose thickness L is 0.70 nm.
    What is the energy of an incident electron whose transmission coefficient is 0.0010?

      \textcolor{hwColor}{
        $
          h=6 ~ eV ~~~~ L=0.70 \times 10^{-9} m ~~~~ T=0.0010 ~~~~ m=9.11 \times 10^{-31} ~~~~ \hbar=1.055 \times 10^{-34} \\
          \\
          T=e^{-2Gl} ~~~~~ G=\dfrac{\sqrt{2m(h-E)}}{\hbar} \\
          \\
          ln(T) = 2\dfrac{\sqrt{2m(h-E)}}{\hbar}L \\
          \\
          \left[\hbar ln(T)\right]^2=8mL^(h-E) \\
          \\
          \left[\hbar ln(T)\right]^2=8mL^2h-8mL^2E \Longrightarrow E=\dfrac{8mL^2h-\left[\hbar ln(T)\right]^2}{8(9.11 \times 10^{-31})L^2} \\
          \\
          =\dfrac{8(9.11 \times 10^{-31})(0.70 \times 10^{-9})^2(6)-\left[(1.055 \times 10^{-34}) ln(0.0010)\right]^2}{8(9.11 \times 10^{-31})(0.70 \times 10^{-9})^2} \\
          \\
          \\
          \Longrightarrow E=6~ eV
        $
      }

    \item (68) (a) Suppose a beam of 5.0 eV protons strikes a potential energy barrier of height 6.0 eV and thickness 0.70 nm, at a rate equivalent to a current of 1000 A. How long would you have to
    wait—on average—for one proton to be transmitted? (b) How long would you have to wait if the beam consisted of electrons rather than protons?

    \textcolor{hwColor}{
      (a): \\
      $
        n=\dfrac{I}{e}=\dfrac{1000 C/s}{1.60 \times 10^{-19} C} \rightarrow n=6.25 \times 10^{21} ~~ 1/s \\
        \\
        N_T=TN_0 ~~~~ N_T=1 \rightarrow Tnt=1 \\
        \\
        T \approx e^{-2bL} ~~~~~~ b=\sqrt{\dfrac{8 \pi^2 m(U-E)}{h^2}} ~~~~~~ t=\dfrac{e^{2bL}}{n} \\
        \\
        b=\sqrt{\dfrac{8 \pi^2 (1.67 \times 10^{-27})(1.602 \times 10^{-19})}{(6.626 \times 10^{-34})^2}}=2.1934 \times 10^11 ~~~ 1/m \\
        \\
        t=\dfrac{e^{2bL}}{n}=\dfrac{e^{2(2.1934 \times 10^{11})(0.70 \times 10^{-9})}}{6.25 \times 10^21} \Longrightarrow t=1.18 \times 10^{104} ~~ y \\
        \\
      $
    }

    \textcolor{hwColor}{
      (b): \\
      $
        b=\sqrt{\dfrac{8 \pi^2 (9.11 \times 10^{-31})(1.602 \times 10^{-19})}{(6.626 \times 10^{-34})^2}}=5.1232 \times 10^9 ~~ 1/m \\
        \\
        t=\dfrac{e^{2(5.1232 \times 10^9)(0.70 \times 10^{-9})}}{2.1 \times 10^{-19}} ~~ s
      $
    }

    \item (81) Imagine playing baseball in a universe (not ours!) where the Planck constant is 0.60 J s and thus quantum physics affects macroscopic objects. What would be the uncertainty in the
    position of a 0.50 kg baseball that is moving at 20 m/s along an axis
    if the uncertainty in the speed is 1.0 m/s?

      \textcolor{hwColor}{
       $
        \Delta p=m \Delta v=(0.50 ~ kg)(1 ~ m/s)=0.50 ~ kgm/s \\
        \\
        \Delta x=\dfrac{\hbar}{\Delta p}=\dfrac{0.60}{2 \pi(0.50)} \\
        \\
        \Delta p=0.19 ~ m
       $
      }

  \end{enumerate}

\end{document}
