\documentclass[fleqn]{article}
\oddsidemargin 0.0in
\textwidth 6.0in
\thispagestyle{empty}
\usepackage{import}
\usepackage{amsmath}
\usepackage{graphicx}
\usepackage[english]{babel}
\usepackage[utf8x]{inputenc}
\usepackage{float}
\usepackage[colorinlistoftodos]{todonotes}

\definecolor{hwColor}{HTML}{AD53BA}

\begin{document}

  \begin{titlepage}

    \newcommand{\HRule}{\rule{\linewidth}{0.5mm}} % Defines a new command for the horizontal lines, change thickness here

    \center % Center everything on the page
    


    \textsc{\LARGE Arizona State University}\\[1.5cm] % Name of your university/college

    \textsc{\LARGE Physics III }\\[1.5cm] % Major heading such as course name


    \begin{figure}
      \includegraphics[width=\linewidth]{asu.png}
    \end{figure}


    \HRule \\[0.4cm]
    { \huge \bfseries Homework 10}\\[0.4cm] 
    \HRule \\[1.5cm]
    
    \textbf{Behnam Amiri}

    \bigbreak

    \textbf{Prof: David Smith}

    \bigbreak


    \textbf{{\large \today}\\[2cm]}

    \vfill % Fill the rest of the page with whitespace

  \end{titlepage}

  \begin{enumerate}
    \item (1) An electron in a hydrogen atom is in a state with $l=5$. What is the minimum possible value of the semiclassical angle between $\overrightarrow{L}$ and $L_z$
    \item (5) (a) How many $l$ values are associated with $n=3$? (b) How many $m_l$ values are associated with $l=1$?
    \item (9) An electron is in a state with $l=3$ . (a) What multiple of $h$ gives the magnitude of $\overrightarrow{L}$ ? (b) What multiple of $\mu_B$ gives the magnitude of $\overrightarrow{\mu}$? (c) What is the largest possible value of $m_l$, (d) what multiple of $h$ gives the corresponding value of $L_z$, and (e) what multiple of $\mu_B$ gives the corresponding value of $\mu_{orb,z}$? (f) What is the value of the semiclassical angle $\theta$ between the directions of $L_z$ and $\overrightarrow{L}$ What is the value of angle $\theta$ for (g) the second largest possible value of $m_l$ and (h) the smallest (that is, most negative) possible value of $m_l$?
    \item (10) An electron is in a state with n  3.What are (a) the number of possible values of , (b) the number of possible values of , (c) the number of possible values of ms, (d) the number of states in the n=3 shell, and (e) the number of subshells in the n=3 shell?
    \item (12) A magnetic field is applied to a freely floating uniform
    iron sphere with radius The sphere initially had no
    net magnetic moment, but the field aligns $12\%$ of the magnetic moments of the atoms (that is, $12\%$ of the magnetic moments of the
    loosely bound electrons in the sphere, with one such electron per
    atom). The magnetic moment of those aligned electrons is the
    sphere’s intrinsic magnetic moment . What is the sphere’s resulting angular speed $\omega$?
    \item (16) Assume that in the Stern–Gerlach experiment as described for
    neutral silver atoms, the magnetic field has a magnitude of 0.50 T.
    (a) What is the energy difference between the magnetic moment orientations of the silver atoms in the two subbeams? (b) What is the
    frequency of the radiation that would induce a transition between
    these two states? (c) What is the wavelength of this radiation, and (d)
    to what part of the electromagnetic spectrum does it belong?
    \item (23) A cubical box of widths $L_x=L_y=L_z=L$ contains eight electrons. What multiple of h2/8mL2 gives the energy of the ground state of this system? Assume that the electrons do not interact with one another, and do not neglect spin
  \end{enumerate}

\end{document}
