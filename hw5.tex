\documentclass[fleqn]{article}
\oddsidemargin 0.0in
\textwidth 6.0in
\thispagestyle{empty}
\usepackage{import}
\usepackage{amsmath}
\usepackage{graphicx}
\usepackage[english]{babel}
\usepackage[utf8x]{inputenc}
\usepackage{float}
\usepackage[colorinlistoftodos]{todonotes}

\definecolor{hwColor}{HTML}{AD53BA}

\begin{document}

  \begin{titlepage}

    \newcommand{\HRule}{\rule{\linewidth}{0.5mm}} % Defines a new command for the horizontal lines, change thickness here

    \center % Center everything on the page
    

    \textsc{\LARGE Arizona State University}\\[1.5cm] % Name of your university/college

    \textsc{\LARGE Physics III }\\[1.5cm] % Major heading such as course name


    \begin{figure}
      \includegraphics[width=\linewidth]{asu.png}
    \end{figure}


    \HRule \\[0.4cm]
    { \huge \bfseries Homework 5}\\[0.4cm] 
    \HRule \\[1.5cm]
    
    \textbf{Behnam Amiri}

    \bigbreak

    \textbf{Prof: David Smith}

    \bigbreak


    \textbf{{\large \today}\\[2cm]}

    \vfill % Fill the rest of the page with whitespace

  \end{titlepage}

  \begin{enumerate}
    \item Light in vacuum is incident on the surface of a glass slab. In the vacuum the beam makes an angle of 32.0° with the normal to the surface, while in the glass it makes an angle of 21.0° with the normal. What is the index of refraction of the glass?

    \textcolor{hwColor}{
      $
        n_1 sin(\theta_1)=n_2 sin(\theta_2)
      $
    }

    \textcolor{hwColor}{
      $
        n_1 sin(32^\circ)=n_2 sin(21^\circ)
      $
    }

    \textcolor{hwColor}{
      $
        \Longrightarrow n_2=1.4787^\circ
      $
    }

    \bigbreak

    \item In Fig. 33-55, a 2.00-m-long vertical pole extends from the bottom of a swimming pool to a point 50.0 cm above the water. Sunlight is incident at angle $\theta=55.0^\circ$ . What is the length of the shadow of the pole on the level bottom of the pool?
    
    \textcolor{hwColor}{
      $
        tan(55^\circ)=\dfrac{0.50}{d_1}
        \Longrightarrow d_1=0.35 m
      $
    }

    \textcolor{hwColor}{
      $
        \theta_i=90^\circ-55^\circ=35^\circ
      $
    }

    \textcolor{hwColor}{
      $
        n_i sin(\theta)=n_r sin(r)
      $
    }

    \textcolor{hwColor}{
      $
        sin(35^\circ)=1.33 sin(r)
        \Longrightarrow r=25.54^\circ
      $
    }

    \textcolor{hwColor}{
      $
        tan(r)=\dfrac{d_2}{2-0.50}
        \Longrightarrow d_2=0.71 m
      $
    }

    
    \textcolor{hwColor}{
      $
        L=d_1+d_2=1.06 m
      $
    }

    \bigbreak

    \item You look through a camera toward an image of a hummingbird in a plane mirror. The camera is 4.30 m in front of the mirror. The bird is at camera level, 5.00 m to your right and 3.30 m from the mirror. What is the distance between the camera and the apparent position of the bird’s image in the mirror?

    \textcolor{hwColor}{
      $d_1=$ The distance between the camera and the mirror
    }

    \textcolor{hwColor}{
      $d_2=$ The distance between the bird and the mirror
    }

    \textcolor{hwColor}{
      $
        d=\sqrt{(d_1+d_2)^2+5.00^2}
      $
    }

    \textcolor{hwColor}{
      $
        d=\sqrt{(4.30+3.30)^2+5.00^2}
      $
    }

    \textcolor{hwColor}{
      $
        d\approx 9.10 m
      $
    }

    \bigbreak

    \item A moth at about eye level is 10 cm in front of a plane mirror; you are behind the moth, 30 cm from the mirror. What is the distance between your eyes and the apparent position of the moth’s image in the mirror?
    
    \textcolor{hwColor}{
      $
        Distance=10+30= 0.40 cm
      $
    }
  \end{enumerate}

\end{document}
