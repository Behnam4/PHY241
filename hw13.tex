\documentclass[fleqn]{article}
\oddsidemargin 0.0in
\textwidth 6.0in
\thispagestyle{empty}
\usepackage{import}
\usepackage{amsmath}
\usepackage{graphicx}
\usepackage[english]{babel}
\usepackage[utf8x]{inputenc}
\usepackage{float}
\usepackage[colorinlistoftodos]{todonotes}

\definecolor{hwColor}{HTML}{AD53BA}

\begin{document}

  \begin{titlepage}

    \newcommand{\HRule}{\rule{\linewidth}{0.5mm}} % Defines a new command for the horizontal lines, change thickness here

    \center % Center everything on the page



    \textsc{\LARGE Arizona State University}\\[1.5cm] % Name of your university/college

    \textsc{\LARGE Physics III }\\[1.5cm] % Major heading such as course name


    \begin{figure}
      \includegraphics[width=\linewidth]{asu.png}
    \end{figure}


    \HRule \\[0.4cm]
    { \huge \bfseries Homework 13}\\[0.4cm] 
    \HRule \\[1.5cm]

    \textbf{Behnam Amiri}

    \bigbreak

    \textbf{Prof: David Smith}

    \bigbreak


    \textbf{{\large \today}\\[2cm]}

    \vfill % Fill the rest of the page with whitespace

  \end{titlepage}

  \begin{enumerate}
    \item (3) The ground-state energy of an electron trapped in a onedimensional infinite potential well is 2.6 eV. What will this quantity
    be if the width of the potential well is doubled?

      \textcolor{hwColor}{
        Length of the well L with ground state energy E, and when the length change to $L^{'}$, the the ground state enegry is $E^{'}$ \\
        $
          E_n=\left(\dfrac{h^2}{8mL^2}\right)n^2 \\
          \\
          \dfrac{(L^{'})^2}{L^2}=\dfrac{E}{E^{'}} \rightarrow E^{'}=\dfrac{(L^{'})^2}{L^2} E \\
          \\
          \dfrac{(L^{'})^2}{L^2}=2 \rightarrow \dfrac{L^2}{(L^{'})^2}=\dfrac{1}{4} \longrightarrow E^{'}=\dfrac{E}{4} \\
          \\
          E^{'}=\dfrac{2.6 ~~ eV}{4}=0.65 ~~ eV
        $
      }

    \item (5) What must be the width of a one-dimensional infinite potential well if an electron trapped in it in the n=3 state is to
    have an energy of 4.7 eV?

      \textcolor{hwColor}{
        $
          E_n=\left(\dfrac{h^2}{8mL^2}\right)n^2=\dfrac{(nhc)^2}{8(mc^2)L^2} \\
          \\
          \\
          L=\dfrac{nhc}{\sqrt{8(mc^2)E_n}}=\dfrac{3 \times 1240 ~~ eV nm}{\sqrt{8(511 \times 10^3)4.7 ~~ eV}} \\
          \\
          L=0.85 ~~ nm
        $
      }

    \item (9) Suppose that an electron trapped in a one-dimensional infinite well of width 250 pm is excited from its first excited state to its third
    excited state. (a) What energy must be transferred to the electron for this quantum jump? The electron then de-excites back to its ground state by emitting light. In the various possible ways it can do this,
    what are the (b) shortest, (c) second shortest, (d) longest, and (e) second longest wavelengths that can be emitted? (f) Show the various possible ways on an energy-level diagram. If light of wavelength
    29.4 nm happens to be emitted, what are the (g) longest and (h) shortest wavelength that can be emitted afterwards?
    
      \textcolor{hwColor}{
        A: \\
        $
          E_n=\left(\dfrac{h^2}{8mL^2}\right)n^2 \\
          \\
          \Delta E_{2 \rightarrow 4}\left(\dfrac{h^2}{8mL^2}\right)(n^2_4-n^2_2)=\left(\dfrac{6.63 \times 10^{-34}}{8(9.11 \times 10^{-31})(250 \times 10^{-12})^2}\right)(4^2-2^2) \\
          \\
          \Delta E_{2 \rightarrow 4}=72.2 ~~ eV
        $
      }

      \textcolor{hwColor}{
        B: \\
        $
          \lambda=\dfrac{hc}{\Delta E}=\dfrac{hc}{E^{'}-E^{''}} \\
          \\
          E^{'}-E^{''}=\dfrac{h^2}{8mL^2}((n^{'})^2-(n^{''})^2) \\
          \\
          \lambda=\dfrac{8mcL^2}{h(((n^{'})^2-(n^{''})^2))} \\
          \\
          \Delta E_{4 \rightarrow 1}=\dfrac{8(9.11 \times 10^{-31})(3 \times 10^8)(250 \times 10^{-12})}{(6.63 \times 10^{-34})(4^2-2^2)} \\
          \\
          \Delta E_{4 \rightarrow 1}=1.37 \times 10^{-8} ~~ m
        $
      }

      \textcolor{hwColor}{
        C: \\
        $
          \Delta E_{4 \rightarrow 2}=\dfrac{8(9.11 \times 10^{-31})(3 \times 10^8)(250 \times 10^{-12})}{(6.63 \times 10^{-34})(4^2-2^2)} \\
          \\
          \Delta E_{4 \rightarrow 2}=1.72 \times 10^{-8} ~~ m
        $
      }

      \textcolor{hwColor}{
        D: \\
        $
          \Delta E_{2 \rightarrow 1}=\dfrac{8(9.11 \times 10^{-31})(3 \times 10^8)(250 \times 10^{-12})}{(6.63 \times 10^{-34})(2^2-1^2)} \\
          \\
          \Delta E_{2 \rightarrow 1}=6.88 \times 10^{-8} ~~ m
        $
      }


      \textcolor{hwColor}{
        E: \\
        $
          \Delta E_{3 \rightarrow 2}=\dfrac{8(9.11 \times 10^{-31})(3 \times 10^8)(250 \times 10^{-12})}{(6.63 \times 10^{-34})(2^2-1^2)} \\
          \\
          \Delta E_{3 \rightarrow 2}=4.12 \times 10^{-8} ~~ m
        $
      }

      \textcolor{hwColor}{
        G: \\
        $
          \Delta E_{4 \rightarrow 3}=\dfrac{8(9.11 \times 10^{-31})(3 \times 10^8)(250 \times 10^{-12})}{(6.63 \times 10^{-34})(2^2-1^2)} \\
          \\
          \Delta E_{4 \rightarrow 3}=2.94 \times 10^{-8} ~~ m \\
          \\
          \\
          \Delta E_{2 \rightarrow 1}=6.87 \times 10^{-8} ~~ m \\
        $
      }

      \textcolor{hwColor}{
        H: \\
        $
          \Delta E_{3 \rightarrow 2}=4.12 \times 10^{-8} ~~ m \\
        $
      }


    \item (17) An electron in the n=2 state in the finite potential well of
    Fig. 39-7 absorbs 400 eV of energy from an external source. Using the energy-level diagram of Fig. 39-9, determine the electron’s kinetic energy after this absorption, assuming that the electron
    moves to a position for which x=L.

      \textcolor{hwColor}{
        $
          K_e=E_i+E_o-d
        $ where d is the depth of the well \\
        \\
        $
          K_e=109 ~ eV+400~ eV-450~ eV \\
          \\
          K_e=59 ~~ eV
        $
      }

    \item (18) Figure 39-9 gives the energy levels for an electron trapped in a finite potential energy well 450 eV deep. If the electron is in the
    n=3 state, what is its kinetic energy?
    
      \textcolor{hwColor}{
        $
          E_3=K+U=280+0=280 ~~ eV
        $
      }

    \item (22) An electron is contained in the rectangular corral of Fig. 39-13, with widths Lx=800 pm and Ly=1600 pm. What is the
    electron’s ground-state energy?
    
      \textcolor{hwColor}{
        The energy of an electron in two dimensional rectangular: \\
        \\
        $
          E=\dfrac{\pi^2 h^2}{2m}\left((\dfrac{n_x}{L_x})^2+(\dfrac{n_y}{L_y})^2\right) \\
          \\
          =\dfrac{\pi^2 h^2}{2m}\left((\dfrac{n_x}{L_x})^2+(\dfrac{n_y}{L_y})^2\right)=\dfrac{(6.63 \times 10^{-34})^2}{8(9.11 \times 10^{-31})}\left((\dfrac{1}{608 \times 10^{-12}})^2+(\dfrac{1}{1880 \times 10^{-12}})^2\right) \\
          \\
          E=1.125 ~ eV
        $
      }

    \item (23) An electron is contained in the rectangular box of Fig. 39-14, with widths Lx=800 pm, Ly=1600 pm, and Lz=390 pm. What
    is the electron’s ground-state energy?

      \textcolor{hwColor}{
        $
          E=E_x+E_y+E_Z=\dfrac{h^2}{8m}\left[\dfrac{n^2_x}{L^2_x}+\dfrac{n^2_y}{L^2_y}+\dfrac{n^2_z}{L^2_z}\right] \\
          \\
          E=\dfrac{6.63 \times 10^{-34}}{8(9.11 \times 10^{-31})}\left[\dfrac{1}{800 \times 10^{-12}}+\dfrac{1}{1600 \times 10^{-12}}+\dfrac{1}{390 \times 10^{-12}}\right] \\
          \\
          E=2.52 \times 10^{24} ~~ eV
        $
      }

    \item (28) A cubical box of widths Lx=Ly=Lz=L contains an electron. What multiple of h2/8mL2 , where m is the electron mass, is
    (a) the energy of the electron’s ground state, (b) the energy of its second excited state, and (c) the difference between the energies of its
    second and third excited states? How many degenerate states have the energy of (d) the first excited state and (e) the fifth excited state?

      \textcolor{hwColor}{
        A: \\
        $
          E=\dfrac{h^2}{8m}\left[\dfrac{n^2_x}{L^2_x}+\dfrac{n^2_y}{L^2_y}+\dfrac{n^2_z}{L^2_z}\right]=\dfrac{h^2}{8mL^2}(n^2_x+n^2_y+n^2_z) \\
          \\
          E_G=\dfrac{h^2}{8mL^2}(1+1+1)=3\dfrac{h^2}{8mL^2} \\
        $ \\
      }

      \textcolor{hwColor}{
        B: \\
        $
          E_2=\dfrac{h^2}{8mL^2}(4+4+1)=9\dfrac{h^2}{8mL^2} \\
        $ \\
      }

      \textcolor{hwColor}{
        C: \\
        $
          E_3=\dfrac{h^2}{8mL^2}(1+1+9)=9\dfrac{h^2}{8mL^2} \\
          \\
          E_3-E_2=2\dfrac{h^2}{8mL^2}
        $ \\
      }

      \textcolor{hwColor}{
        D: \\
        $
          (n_x, n_y, n_z)=(2,1,1) ~~~~ (n_x, n_y, n_z)=(1,2,1) ~~~~ (n_x, n_y, n_z)=(1,1,2) ~~~~
        $ \\
        The degeneracy is three. 
      }

      \textcolor{hwColor}{
        E: \\
        $
          (n_x, n_y, n_z)=(1,2,3), (1,3,2), (2,3,1), (2,1,3), (3,1,2), (3,2,1)
        $ \\
        The degeneracy is six. 
      }

    \item (38) An atom (not a hydrogen atom) absorbs a photon whose associated frequency is $6.2 \times 10^{14}$ Hz. By what amount does the energy of the atom increase?

      \textcolor{hwColor}{
        $
         \Delta E=hf=(4.14 \times 10^{-15})(6.2 \times 10^{14}) \\
         \\
         \Delta E=2.6 ~~ eV
        $
      }

    \item (42) A hydrogen atom, initially at rest in the n=4 quantum state, undergoes a transition to the ground state, emitting a photon
    in the process. What is the speed of the recoiling hydrogen atom?
    (Hint:This is similar to the explosions of Chapter 9.)

      \textcolor{hwColor}{
        $
          P_{ph}=\dfrac{E}{c}=mv=\dfrac{hf}{c} \\
          \\
          v=\dfrac{A}{mc}\left[\dfrac{1}{n^2_1}+\dfrac{1}{n^2_2}\right]=\dfrac{13.6 \times 1.602 \times 10^{-19}}{(1.67 \times 10^{-27})(2.998 \times 10^{8})}(\dfrac{1}{1}-\dfrac{1}{16}) \\
          \\
          v=4.1 ~~ m/s
        $
      }

    \item (48) Light of wavelength 121.6 nm is emitted by a hydrogen atom. What are the (a) higher quantum number and (b) lower
    quantum number of the transition producing this emission? (c) What is the name of the series that includes the transition?

      \textcolor{hwColor}{
        A: \\
        $
          E_{ph}=A\left[\dfrac{1}{n^2_2}-\dfrac{1}{n^2_1}\right] \\
          \\
          \dfrac{E_{ph}}{hc}=\dfrac{A}{hc}\left[\dfrac{1}{n^2_2}-\dfrac{1}{n^2_1}\right] \rightarrow \dfrac{hc}{\lambda A}=\dfrac{1}{n^2_2}-\dfrac{1}{n^2_1} \\
          \\
          \dfrac{(6.63 \times 10^{-34})(3 \times 10^8)}{(102.6 \times 10^{-9})(13.6 \times 1.602 \times 10^{-19})}=\dfrac{1}{n^2_2}-\dfrac{1}{n^2_1} \\
          \\
          \Rightarrow \dfrac{1}{n^2_2}-\dfrac{1}{n^2_1}=0.9 \Longrightarrow n_1=3 \\
        $
      }

      \textcolor{hwColor}{
        B: \\
        $
          n_2=1 \\
        $
      }

      \textcolor{hwColor}{
        C: \\
        $
          Lyman \\
        $
      }



  \end{enumerate}

\end{document}
