\documentclass[fleqn]{article}
\oddsidemargin 0.0in
\textwidth 6.0in
\thispagestyle{empty}
\usepackage{import}
\usepackage{amsmath}
\usepackage{graphicx}
\usepackage[english]{babel}
\usepackage[utf8x]{inputenc}
\usepackage{float}
\usepackage[colorinlistoftodos]{todonotes}

\definecolor{hwColor}{HTML}{AD53BA}

\begin{document}

  \begin{titlepage}

    \newcommand{\HRule}{\rule{\linewidth}{0.5mm}} % Defines a new command for the horizontal lines, change thickness here

    \center % Center everything on the page
    


    \textsc{\LARGE Arizona State University}\\[1.5cm] % Name of your university/college

    \textsc{\LARGE Physics III }\\[1.5cm] % Major heading such as course name


    \begin{figure}
      \includegraphics[width=\linewidth]{asu.png}
    \end{figure}


    \HRule \\[0.4cm]
    { \huge \bfseries Homework 8}\\[0.4cm] 
    \HRule \\[1.5cm]
    
    \textbf{Behnam Amiri}

    \bigbreak

    \textbf{Prof: David Smith}

    \bigbreak


    \textbf{{\large \today}\\[2cm]}

    \vfill % Fill the rest of the page with whitespace

  \end{titlepage}

  \begin{enumerate}
    \item (1) The distance between the first and fifth minima of a
    single-slit diffraction pattern is 0.35 mm with the screen 40 cm
    away from the slit, when light of wavelength 550 nm is used. (a)
    Find the slit width. (b) Calculate the angle u of the first diffraction
    minimum.

      \textcolor{hwColor}{
        $
          a~sin(\theta)=m \lambda \\
          sin(\theta)=\dfrac{y}{d} \\
          \\
          y=\dfrac{m \lambda d}{a} \\
          \\
          \Delta y=\dfrac{\lambda d \Delta m}{a} \\
          \\
          a=\dfrac{(550 \times 10^{-9})(0.40)(5-1)}{0.35 \times 10^{-3}} \Longrightarrow a=2.5 \times 10^{-3} ~~ m
          \\
          \theta=arcsin(\dfrac{m\lambda}{a})=arcsin(\dfrac{550 \times 10^-9}{2.5 \times 10^-3}) \Longrightarrow \theta=2.2 \times 10^-4 ~~ rad
        $
      }
    
    \item (6) Monochromatic light of wavelength 441 nm is incident on a
    narrow slit. On a screen 2.00 m away, the distance between the second diffraction minimum and the central maximum is 1.50 cm. (a)
    Calculate the angle of diffraction $\theta$ of the second minimum. (b)
    Find the width of the slit.

      \textcolor{hwColor}{
        $
          a~sin(\theta)=m \lambda \\
          sin(\theta)=\dfrac{y}{d} \\
          \\
          y=\dfrac{m \lambda d}{a} \\
          \\
          \theta=arcsin(\dfrac{1.50 \times 10^{-2}}{2}) \Longrightarrow \theta=0.43^{\circ} \\
          \\
          a=\dfrac{4(441 \times 10^{-9})}{1.50 \times 10^{-2}} \Longrightarrow a=1.18 \times 10^{-4}
        $
      }

    \item (13) Monochromatic light with wavelength 538 nm is incident on
    a slit with width 0.025 mm. The distance from the slit to a screen is
    3.5 m. Consider a point on the screen 1.1 cm from the central maximum. Calculate (a) $\theta$ for that point, (b) $\alpha$, and (c) the ratio of the
    intensity at that point to the intensity at the central maximum.

      \textcolor{hwColor}{
        $
          sin(\theta)=\dfrac{y}{L}=\dfrac{0.011}{3.5} \\
          \theta=sin^{-1}(\dfrac{0.011}{3.5})=0.18^{\circ} \\
          \\
          \alpha=\pi a\dfrac{sin(\theta)}{\lambda}=\pi \dfrac{sin(0.18^{\circ})}{(538 \times 10^{-9})}=0.46 ~rad \\
          \\
          \dfrac{I}{I_m}=\left(\dfrac{sin(\alpha)}{\alpha}\right)=\left(\dfrac{sin(0.46)}{0.46}\right)=0.94
        $
      }

    \item (20) The radar system of a navy cruiser transmits at a wavelength
    of 1.6 cm, from a circular antenna with a diameter of 2.3 m. At a
    range of 6.2 km, what is the smallest distance that two speedboats
    can be from each other and still be resolved as two separate objects
    by the radar system?

      \textcolor{hwColor}{
        $
          sin(\theta)=1.22\dfrac{\lambda}{d} \\
          \\
          L\theta=D \rightarrow \theta=\dfrac{D}{L} \\
          \\
          \dfrac{D}{L}=1.22\dfrac{L\lambda}{d} \\
          \\
          D=1.22\dfrac{(6.2 \times 10^3)(1.6 \times 10^{-2})}{2.3} \\
          \\ 
          D=53 ~ m
        $
      }

    \item (25) Find the separation of two points on the Moon’s surface
    that can just be resolved by the 200 in. ( =5.1 m) telescope at
    Mount Palomar, assuming that this separation is determined by
    diffraction effects. The distance from Earth to the Moon is $3.8 \times 10^5$ km.Assume a wavelength of 550 nm for the light.

      \textcolor{hwColor}{
        $
          sin(\theta)=1.22\dfrac{\lambda}{d} \\
          \\
          L\theta=D \rightarrow \theta=\dfrac{D}{L} \\
          \\
          \dfrac{D}{L}=1.22\dfrac{L\lambda}{d} \\
          \\
          D=1.22\dfrac{(3.8 \times 10^8)(550 \times 10^{-9})}{5.1} \\
          \\ 
          D=50 ~ m
        $
      }

    
    \item (39) Light of wavelength 440 nm passes through a double slit,
    yielding a diffraction pattern whose graph of intensity I versus angular position $\theta$ is shown in Fig. 36-44. Calculate (a) the slit width
    and (b) the slit separation. (c) Verify the displayed intensities of
    the $m_1$ and $m_2$ interference fringes.

      \textcolor{hwColor}{
        $
         a=\dfrac{\lambda}{sin(\theta)}=\dfrac{0.440}{sin(5)}=5.05 \mu m \\
         \\
         d=4a=4(5.05)=20.s \mu m \\
         \\
         \alpha=\dfrac{\pi a sin(\theta)}{\pi}=\dfrac{\pi (5.05)(sin(1.25))}{0.440}=0.787 ~~rad \\
         \\
         I=I_m=(\dfrac{sin(\alpha)}{\alpha})^2=(7)(\dfrac{sin(0.787)}{0.787})=5.7 ~~\dfrac{mW}{cm^2} \\
         \\
         m=2 \rightarrow I=2.9 ~~\dfrac{mW}{cm^2}
        $
      }

    \vfill

    \item (42) (a) In a double-slit experiment, what ratio of d to a
    causes diffraction to eliminate the fourth bright side fringe? (b)
    What other bright fringes are also eliminated?

    \textcolor{hwColor}{
      $
        a sin(\theta)=\lambda \\
        \\
        d sin(\theta)=m_2 \lambda \\
        \\
        m_2=\dfrac{d}{a} \\
        m=4 \\
        \\
        d=4\lambda \\
        \\
        asin(\theta)=m_1 \lambda \\
        \\
        \dfrac{m_2}{m_1}=\dfrac{d}{a} \\ 
        \\
        m_2=4m_1 \\
        \\
      $
      The missing fringes are 4th, 8th, 12th and... \\
      $
        \\
        \dfrac{d}{a}=\dfrac{4}{m_1}
      $
      We have four ratios to exactly eliminate the fourth maximum.
    }


    \item (49) Light of wavelength 600 nm is incident normally on a diffraction grating. Two adjacent maxima occur at angles given by $sin(\theta)=0.2$ and $sin(\theta)=0.3$. The fourth-order maxima
    are missing. (a) What is the separation between adjacent slits?
    (b) What is the smallest slit width this grating can have? For that
    slit width, what are the (c) largest, (d) second largest, and (e) third
    largest values of the order number m of the maxima produced by
    the grating?

      \textcolor{hwColor}{
        $
          dsin(\theta)=m\lambda \\
          \\
          dsin(\theta_1)=m \lambda ~~~~~ dsin(\theta_2)=(m+1)\lambda \\
          \\
          d\left(sin(\theta_2)-sin(\theta_1)=\lambda\right) \\
          \\
          d=\dfrac{\lambda}{sin(\theta_2)-sin(\theta_1)} \\ 
          \\
          d=\dfrac{600 \times 10^{-9}}{0.3-0.2} \rightarrow d=6 \times 10^{-6} ~~m \\
          \\
          \\
          asin(\theta)=m\lambda \\
          a=\dfrac{d}{4} \\
          \\
          a=\dfrac{6 \times 10^{-6}}{4} \rightarrow a=1.5 \times 10^{-6} ~~m \\
          \\
          \\
          m<\dfrac{d}{\lambda} \rightarrow m<\dfrac{6 \times 10^{-6}}{600 \times 10^{-9}} \\
          \\
          m<10
        $
        Hence, the largest value for m is 9 \\
        The second largest number for m is 7 \\
        And the third largest number for m is 6
      }

    \item (60) The D line in the spectrum of sodium is a doublet with wavelengths 589.0 and 589.6 nm. Calculate the minimum number of lines
    needed in a grating that will resolve this doublet in the secondorder spectrum.

    \textcolor{hwColor}{
      $
        R=\dfrac{\lambda}{\Delta \lambda}=Nm \\
        \\
        N=\dfrac{\lambda}{m \Delta \lambda} \\
        \\
        N= \dfrac{\dfrac{589.6+589}{2}}{2(589.6-589)} \\
        \\
        \Longrightarrow N=491
      $
    }

  \end{enumerate}

\end{document}
