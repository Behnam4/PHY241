\documentclass[fleqn]{article}
\oddsidemargin 0.0in
\textwidth 6.0in
\thispagestyle{empty}
\usepackage{import}
\usepackage{amsmath}
\usepackage{graphicx}
\usepackage[english]{babel}
\usepackage[utf8x]{inputenc}
\usepackage{float}
\usepackage[colorinlistoftodos]{todonotes}

\definecolor{hwColor}{HTML}{AD53BA}

\begin{document}

  \begin{titlepage}

    \newcommand{\HRule}{\rule{\linewidth}{0.5mm}} % Defines a new command for the horizontal lines, change thickness here

    \center % Center everything on the page
    


    \textsc{\LARGE Arizona State University}\\[1.5cm] % Name of your university/college

    \textsc{\LARGE Physics III }\\[1.5cm] % Major heading such as course name


    \begin{figure}
      \includegraphics[width=\linewidth]{asu.png}
    \end{figure}


    \HRule \\[0.4cm]
    { \huge \bfseries Homework 11}\\[0.4cm] 
    \HRule \\[1.5cm]
    
    \textbf{Behnam Amiri}

    \bigbreak

    \textbf{Prof: David Smith}

    \bigbreak


    \textbf{{\large \today}\\[2cm]}

    \vfill % Fill the rest of the page with whitespace

  \end{titlepage}

  \textbf{Chapter 37}
  \begin{enumerate}
    \item (40) How much work must be done to increase the speed of an electron from rest to (a) 0.500c, (b) 0.990c, and (c) 0.9990c?

      \textcolor{hwColor}{
        A: \\
        $
          K=mc^2(\dfrac{1}{\sqrt{1-B^2}}-1)=(9.109e-31)(3e8)^2(\dfrac{1}{\sqrt{1-0.500^2}}-1) \\
          K=0.0795 ~ MeV
        $
      }
      
      \textcolor{hwColor}{   
        \rule{15cm}{0.4pt}   
      } 

      \textcolor{hwColor}{
        B: \\
        $
          K=mc^2(\dfrac{1}{\sqrt{1-B^2}}-1)=(9.109e-31)(3e8)^2(\dfrac{1}{\sqrt{1-0.990^2}}-1) \\
          K=3.20 ~ MeV
        $
      }

      \textcolor{hwColor}{   
        \rule{15cm}{0.4pt}   
      } 

      \textcolor{hwColor}{
        C: \\
        $
          K=mc^2(\dfrac{1}{\sqrt{1-B^2}}-1)=(9.109e-31)(3e8)^2(\dfrac{1}{\sqrt{1-0.9990^2}}-1) \\
          K=11 ~ MeV
        $
      }
    \item (43) How much work must be done to increase the speed of an electron (a) from 0.18c to 0.19c and (b) from 0.98c to 0.99c? Note that the speed increase is 0.01c in both cases

      \textcolor{hwColor}{
        A: \\
        $
          K_i=mc^2(\dfrac{1}{\sqrt{1-B^2}}-1)=(9.109e-31)(3e8)^2(\dfrac{1}{\sqrt{1-0.18^2}}-1) \\
          K_i=8508.125 ~ eV \\
          \\
          \\
          K_f=mc^2(\dfrac{1}{\sqrt{1-B^2}}-1)=(9.109e-31)(3e8)^2(\dfrac{1}{\sqrt{1-0.19^2}}-1) \\
          K_f=9506.7 ~ eV \\
          \\
          \Delta K=K_f-K_i=1.0 ~ keV
        $
      }

      \textcolor{hwColor}{   
        \rule{15cm}{0.4pt}   
      }

      \textcolor{hwColor}{
        B: \\
        $
          K_i=mc^2(\dfrac{1}{\sqrt{1-B^2}}-1)=(9.109e-31)(3e8)^2(\dfrac{1}{\sqrt{1-0.98^2}}-1) \\
          K_i=2056181.25 ~ eV \\
          \\
          \\
          K_f=mc^2(\dfrac{1}{\sqrt{1-B^2}}-1)=(9.109e-31)(3e8)^2(\dfrac{1}{\sqrt{1-0.99^2}}-1) \\
          K_f=3119793.75 ~ eV \\
          \\
          \Delta K=K_f-K_i=1.1 ~ MeV
        $
      }
        
    \item (45) In a high-energy collision between a cosmic-ray particle and a particle near the top of Earth’s atmosphere, 120 km above sea
    level, a pion is created. The pion has a total energy E of 1.35  105 MeV and is traveling vertically downward. In the pion’s rest frame,
    the pion decays 35.0 ns after its creation. At what altitude above sea level, as measured from Earth’s reference frame, does the decay occur? The rest energy of a pion is 139.6 MeV.

      \textcolor{hwColor}{
        $
          d=c\Delta t, ~~~ \Delta t=\gamma \Delta t_0 \\
          d=\dfrac{cE \Delta t_0}{mc^2}=\dfrac{(3.0 \times 10^8)(1.35 \times 10^5)(35 \ times 10^{-9})}{139.6 MeV} \\
          \\
          d=110 ~ km
        $
      }

    \item (48) The mass of a muon is 207 times the electron mass; the average lifetime of muons at rest is 2.20 ms. In a certain experiment, muons moving through a laboratory are measured to have
    an average lifetime of 6.90 ms. For the moving muons, what are (a)b, (b) K, and (c) p (in MeV/c)?

      \textcolor{hwColor}{
        A: \\
        $
          \Delta t=\gamma \Delta t_0 ~~~ \gamma=\dfrac{1}{\sqrt{1-\beta^2}} \\
          \beta = \sqrt{1-(\dfrac{\Delta t_0}{\Delta t})^2}=\sqrt{1-(\dfrac{2.20 \times 10^{-6}}{6.90 \times 10^{-6}})^2} \\
          \\
          \beta=0.9478
        $
      }

      \textcolor{hwColor}{   
        \rule{15cm}{0.4pt}   
      }

      \textcolor{hwColor}{
        B: Lorentz factor \\
        $
          \gamma= \dfrac{1}{\sqrt{1-0.9478^2}}=3.136 \\
          \\
          \\
          K=mc^2(\gamma -1)=(207 \times 0.511 MeV/c^2)c^2(3.136-1)=226 ~ MeV/c^2 
        $
      }

      \textcolor{hwColor}{   
        \rule{15cm}{0.4pt}   
      }

      \textcolor{hwColor}{
        C: \\
        $
          \rho=\gamma m_\mu v=\gamma m_\mu \beta_c=(3.136)(105.8)(0.9478) \\
          \rho=314 ~ MeV/c
        $
      }

    \item (54) What is b for a particle with (a) K=2.00E0 and (b) E=2.00E0?

      \textcolor{hwColor}{
        A: \\
        $
          K=mc^2(\gamma -1) \Longrightarrow \gamma=\dfrac{K}{mc^2}+1=\dfrac{2.00 mc^2}{mc^2}+1 \\
          \\
          \gamma=3 \\
          \\
          \gamma=\dfrac{1}{\sqrt{1-\beta^2}} \Longrightarrow \beta=\sqrt{1-\dfrac{1}{\gamma^2}}=\sqrt{1-\dfrac{1}{3^2}} \\
          \beta=0.9428
        $
      }

      \textcolor{hwColor}{   
        \rule{15cm}{0.4pt}   
      }

      \textcolor{hwColor}{
        B: \\
        $
        \beta=\sqrt{1-\dfrac{1}{\gamma^2}}=\sqrt{1-\dfrac{1}{2^2}} \\
        \beta=0.866
        $
      }

  \end{enumerate}

  \textbf{Chapter 38}
  \begin{enumerate}
    \item (3) At what rate does the Sun emit photons? For simplicity, assume that the Sun’s entire emission at the rate of 3.9  1026 W is at the single wavelength of 550 nm.

      \textcolor{hwColor}{
        $
          E=hf=\dfrac{hc}{\lambda} ~~~~ h=6.626 \times 10^{-34} J.s \\
          \\
          P=\dfrac{Rhc}{\lambda} ~~~~ R=\dfrac{\lambda P}{hc}=\dfrac{(550 ~ nm)(3.9 \times 10^{26} ~ W)}{(6.63 \times 10^{-34} (2.998 \times 10^8))}=10^45 ~~ Photons/s
        $
      }

    \item (6) The yellow-colored light from a highway sodium lamp is brightest at a wavelength of 589 nm. What is the photon energy for light at that wavelength?

      \textcolor{hwColor}{
        $
          E=\dfrac{hc}{\lambda}=0.033 \times 10^{-17} ~ J
        $
      }

    \item (9) A 100 W sodium lamp $(\lambda=589 nm)$ radiates energy uniformly in all directions. (a) At what rate are photons emitted by the lamp? (b) At what distance from the lamp will a totally absorbing screen absorb photons at the rate of 1.00 photon/cm2 s? (c) What is the photon flux (photons per unit area per unit time) on a small screen 2.00 m from the lamp?

      \textcolor{hwColor}{
        A: \\
        $
          P=RE ~~~ E=hf ~~~ f=\dfrac{c}{\lambda} \\
          \\
          E=\dfrac{hc}{\lambda} \rightarrow R=\dfrac{P \lambda}{hc} \\
          R=\dfrac{100(589 \times 10^{-9})}{6.626 \times 10^{-34}(3 \times 10^8)}=2.98 \times 10^20 ~~ Photons/s
        $
      }

      \textcolor{hwColor}{   
        \rule{15cm}{0.4pt}   
      }

      \textcolor{hwColor}{
        B: \\
        $
          R=AI=4 \pi r^2 I \rightarrow r=\sqrt{\dfrac{R}{4 \pi I}}=\sqrt{\dfrac{2.96 \times 10^20}{4 \pi (10^4)}} \\
          \\
          r=4.86 \times 10^7 ~ m 
        $
      }

      \textcolor{hwColor}{   
        \rule{15cm}{0.4pt}   
      }

      \textcolor{hwColor}{
        C: \\
        $
          I=\dfrac{R}{4 \pi r^2}=\dfrac{2.96 \times 10^20}{16\pi} \\
          \\
          I=5.89 \times 10^18 ~~ Photons/m^2s
        $
      }

    \item (13) A special kind of lightbulb emits monochromatic light of wavelength 630 nm. Electrical energy is supplied to it at the rate of
    60 W, and the bulb is 93\% efficient at converting that energy to
    light energy. How many photons are emitted by the bulb during its
    lifetime of 730 h?

      \textcolor{hwColor}{
        $
          E=0.93Pt \\
          \\
          E_{Ph}=\dfrac{hc}{\lambda} \\
          \\
          N=\dfrac{E}{E_{Ph}}=\dfrac{0.93(60 ~ W)(730 \times 3600 ~ s)(630 \times 10^{-9} ~ m)}{(6.626 \times 10^{-34} ~ js)(2.998 \times 10^8 ~ m/s)} \\
          \\
          N=4.65 \times 10^26 ~ Photon
        $
      }

    \item (18) You wish to pick an element for a photocell that will operate
    via the photoelectric effect with visible light. Which of the following are suitable (work functions are in parentheses): tantalum (4.2
    eV), tungsten (4.5 eV), aluminum (4.2 eV), barium (2.5 eV),
    lithium (2.3 eV)?

    \textcolor{hwColor}{
      $
        E_{Ph}=\dfrac{hc}{\lambda} \\
        \\
        E_{max}=\dfrac{(6.626 \times 10^{-34})(2.998 \times 10^8)}{400 \times 10^{-9}}=3.1 ~ eV \\
        \\
      $
      Barium and Lithium should be good. 
    }

    \item (22) The wavelength associated with the cutoff frequency for
    silver is 325 nm. Find the maximum kinetic energy of electrons
    ejected from a silver surface by ultraviolet light of wavelength
    254 nm

      \textcolor{hwColor}{
        $
          K_{max}=E_{Photon}-\Phi=\dfrac{hc}{\lambda}-\dfrac{hc}{\lambda_{max}} \\
          K_{max}=\dfrac{1240}{254}-\dfrac{1240}{325} \\
          \\
          \Longrightarrow K_{max}=1.07 ~ eV
        $
      }

    \item (24) In a photoelectric experiment using a sodium surface, you
    find a stopping potential of 1.85 V for a wavelength of 300 nm
    and a stopping potential of 0.820 V for a wavelength of 400 nm.
    From these data find (a) a value for the Planck constant, (b) the
    work function for sodium, and (c) the cutoff wavelength l0 for
    sodium

      \textcolor{hwColor}{
        A: \\
        $
          hf=k_{max}+\Phi ~~~~ K_{max}=eV \rightarrow \dfrac{hc}{\lambda}=eV+\Phi \\
          \\
          \dfrac{hc}{\lambda_1}-\dfrac{hc}{\lambda_2}=e(V_1-V_2) \\
          \\
          \Longrightarrow h=\dfrac{e \lambda_2 \lambda_1 (V_1-V_2)}{c(\lambda_2 - \lambda_1)} \\
          \\
          h=\dfrac{(1.60 \times 10^{-19})(400 \times 10^{-9})(300 \times 10^{-9})(1.85~V~-0.85~V )}{(2.998 \times 10^8\left((400 \times 10^{-9})-(300 \times 10^{-9})\right))} \\
          \\
          h=6.404 \times 10^{-34} ~ J.s
        $
      }

    \item (29) What (a) frequency, (b) photon energy, and (c) photon momentum magnitude (in keV/c) are associated with x rays having
    wavelength 35.0 pm?

      \textcolor{hwColor}{
        A: \\
        $
          f=\dfrac{c}{\lambda}=\dfrac{2.998 \times 10^8}{35 \times 10^{-12}} \\
          \\
          f=8.57 \times 10^{18} ~~ Hz
        $
      }

      \textcolor{hwColor}{   
        \rule{15cm}{0.4pt}   
      }

      \textcolor{hwColor}{
        B: \\
        $
          E=hf=(4.136 \times 10^{-15})(8.57 \times 10^{18})=3.55 \times 10^4 ~~ eV
        $
      }

      \textcolor{hwColor}{   
        \rule{15cm}{0.4pt}   
      }

      \textcolor{hwColor}{
        C: \\
        $
          \rho=\dfrac{h}{\lambda}=\dfrac{(4.136 \times 10^{-15})(2.998 \times 10^8)}{35 \times 10^{-12}} \\
          \\
          \rho=35.4 keV/c
        $
      }

    \item (36) Gamma rays of photon energy 0.511 MeV are directed onto
    an aluminum target and are scattered in various directions by
    loosely bound electrons there. (a) What is the wavelength of the incident gamma rays? (b) What is the wavelength of gamma rays
    scattered at 90.0° to the incident beam? (c) What is the photon energy of the rays scattered in this direction?

      \textcolor{hwColor}{
        A: \\
        $
          \lambda=\dfrac{hc}{E}=\dfrac{1240 ~ nm.eV}{0.511 ~ MeV}=2.43 \times 10^{-3} ~ nm \\
          \\
          \lambda=2.43 ~ pm
        $
      }

      \textcolor{hwColor}{   
        \rule{15cm}{0.4pt}   
      }

      \textcolor{hwColor}{
        B: \\
        $
          \lambda^\prime=\lambda+\Delta \lambda=\lambda+\dfrac{h}{m_ec}(1-cos \phi)=2.43+(2.43)(1-cos(90^{\circ})) \\
          \\
          \lambda^\prime=4.86 ~ pm
        $
      }

      \textcolor{hwColor}{   
        \rule{15cm}{0.4pt}   
      }

      \textcolor{hwColor}{
        C: \\
        $
          E^\prime=E(\dfrac{\lambda}{\lambda^\prime})=(0.511 ~ MeV)(\dfrac{2.43 ~ pm}{4.86 ~ pm}) \\
          \\
          E^\prime=0.255 ~ MeV
        $
      }

    \item (37) Consider a collision between an x-ray photon of initial energy 50.0 keV and an electron at rest, in which the photon is scattered backward and the electron is knocked forward. (a) What is
    the energy of the back-scattered photon? (b) What is the kinetic
    energy of the electron?

      \textcolor{hwColor}{
        A: \\
        $
          \Delta \lambda=\dfrac{h}{mc}(1-cos(\Phi))=\dfrac{h}{mc}(1-(-1)) \rightarrow \Delta \lambda=\dfrac{2h}{mc} \\
          \\
          \lambda^\prime =\lambda+\Delta \lambda \\
          \\
          E^\prime=\dfrac{hc}{\lambda^\prime}=\dfrac{hc}{\lambda + \Delta \lambda} \\
          \\
          E^\prime =\dfrac{\dfrac{hc}{\lambda}}{1+\dfrac{\Delta \lambda}{hc}} \\
          \\
          E^\prime=\dfrac{E}{1+2\dfrac{E}{mc^2}}=\dfrac{50 keV}{1+2(50 ~ keV)/(511 ~ keV)} \\
          \\
          E^\prime=41.8 ~ keV
        $
      }

      \textcolor{hwColor}{   
        \rule{15cm}{0.4pt}   
      }

      \textcolor{hwColor}{
        B: \\
        $
          k_e=E-E^\prime=50 ~ keV-41.8 ~ keV=8.2 keV \\
          \\
          k_e=8.2 ~ keV
        $
      }

  \end{enumerate}
  

\end{document}
